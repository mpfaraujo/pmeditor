---
tipo: Múltipla Escolha
dificuldade: Média
disciplina: Matemática
assunto: Probabilidade
gabarito: C
tags: [probabilidade, urna, sistema_de_equações]

fonte: concurso
concurso: FUVEST
banca: FUVEST
ano: 2018
numero: 24
---

\question Em uma urna, há bolas amarelas, brancas e vermelhas. Sabe-se que:

I. A probabilidade de retirar uma bola vermelha dessa urna é o dobro da probabilidade de retirar uma bola amarela.

II. Se forem retiradas 4 bolas amarelas dessa urna, a probabilidade de retirar uma bola vermelha passa a ser $\dfrac{1}{2}$.

III. Se forem retiradas 12 bolas vermelhas dessa urna, a probabilidade de retirar uma bola branca passa a ser $\dfrac{1}{2}$.

A quantidade de bolas brancas na urna é

\begin{choices}
\choice 8
\choice 10
\correctchoice 12
\choice 14
\choice 16
\end{choices}

---
tipo: Múltipla Escolha
dificuldade: Média
disciplina: Matemática
assunto: Função composta
gabarito: D
tags: [função_composta, gráfico, função_afim]

fonte: concurso
concurso: FUVEST
banca: FUVEST
ano: 2018
numero: 25
---

\question Sejam $f:\mathbb{R}\to\mathbb{R}$ e $g:\mathbb{R}\to\mathbb{R}$ definidas por
\[
f(x)=\frac{1}{5}x \quad \text{e} \quad g(x)=10x.
\]
O gráfico da função composta $g\circ f$ é:

\begin{choices}
\choice (A)
\choice (B)
\choice (C)
\correctchoice (D)
\choice (E)
\end{choices}

---
tipo: Múltipla Escolha
dificuldade: Difícil
disciplina: Matemática
assunto: Contagem
gabarito: B
tags: [triângulos, combinação, geometria_plana]

fonte: concurso
concurso: FUVEST
banca: FUVEST
ano: 2018
numero: 26
---

\question Doze pontos são assinalados sobre quatro segmentos de reta de forma que três pontos sobre três segmentos distintos nunca são colineares, como na figura.

O número de triângulos distintos que podem ser desenhados com os vértices nos pontos assinalados é

\begin{choices}
\choice 200
\correctchoice 204
\choice 208
\choice 212
\choice 220
\end{choices}

---
tipo: Múltipla Escolha
dificuldade: Difícil
disciplina: Matemática
assunto: Inclusão-exclusão
gabarito: D
tags: [conjuntos, princípio_da_inclusão_exclusão]

fonte: concurso
concurso: FUVEST
banca: FUVEST
ano: 2018
numero: 27
---

\question Dentre os candidatos que fizeram provas de matemática, português e inglês num concurso, 20 obtiveram nota mínima para aprovação nas três disciplinas. Além disso, sabe-se que:

I. 14 não obtiveram nota mínima em matemática;
II. 16 não obtiveram nota mínima em português;
III. 12 não obtiveram nota mínima em inglês;
IV. 5 não obtiveram nota mínima em matemática e em português;
V. 3 não obtiveram nota mínima em matemática e em inglês;
VI. 7 não obtiveram nota mínima em português e em inglês;
VII. 2 não obtiveram nota mínima em português, matemática e inglês.

A quantidade de candidatos que participaram do concurso foi

\begin{choices}
\choice 44
\choice 46
\choice 47
\correctchoice 48
\choice 49
\end{choices}

---
tipo: Múltipla Escolha
dificuldade: Média
disciplina: Matemática
assunto: Domínio e imagem
gabarito: C
tags: [função_logarítmica, domínio, imagem]

fonte: concurso
concurso: FUVEST
banca: FUVEST
ano: 2018
numero: 28
---

\question Sejam $D_f$ e $D_g$ os maiores subconjuntos de $\mathbb{R}$ nos quais estão definidas, respectivamente, as funções reais
\[
f(x)=\frac{1}{\ln x} \quad \text{e} \quad g(x)=\frac{1}{\ln(1-x)}.
\]
Considere, ainda, $I_f$ e $I_g$ as imagens de $f$ e de $g$, respectivamente.

Nessas condições,

\begin{choices}
\choice $D_f=D_g$ e $I_f=I_g$
\choice tanto $D_f$ e $D_g$ quanto $I_f$ e $I_g$ diferem em apenas um ponto
\correctchoice $D_f$ e $D_g$ diferem em apenas um ponto, $I_f$ e $I_g$ diferem em mais de um ponto
\choice $D_f$ e $D_g$ diferem em mais de um ponto, $I_f$ e $I_g$ diferem em apenas um ponto
\choice tanto $D_f$ e $D_g$ quanto $I_f$ e $I_g$ diferem em mais de um ponto
\end{choices}

---
tipo: Múltipla Escolha
dificuldade: Média
disciplina: Matemática
assunto: Geometria
gabarito: B
tags: [ângulos, polígono_estrelado]

fonte: concurso
concurso: FUVEST
banca: FUVEST
ano: 2018
numero: 29
---

\question Prolongando-se os lados de um octógono convexo $ABCDEFGH$, obtém-se um polígono estrelado, conforme a figura.

A soma indicada vale

\begin{choices}
\choice $1800^\circ$
\correctchoice $3600^\circ$
\choice $5400^\circ$
\choice $7200^\circ$
\choice $9000^\circ$
\end{choices}

---
tipo: Múltipla Escolha
dificuldade: Média
disciplina: Matemática
assunto: Função trigonométrica
gabarito: A
tags: [seno, parâmetro, gráfico]

fonte: concurso
concurso: FUVEST
banca: FUVEST
ano: 2018
numero: 30
---

\question Admitindo que a linha pontilhada represente o gráfico da função $f(x)=a\sin x$ e que a linha contínua represente o gráfico da função $g(x)=b\sin(x+c)$, segue que

\begin{choices}
\correctchoice $0<a<1$ e $0<b<1$
\choice $a>1$ e $b<1$
\choice $a=1$ e $b>1$
\choice $0<a<1$ e $b>1$
\choice $0<a<1$ e $b=1$
\end{choices}

---
tipo: Múltipla Escolha
dificuldade: Média
disciplina: Matemática
assunto: Juros compostos
gabarito: E
tags: [juros_compostos, aplicação_financeira]

fonte: concurso
concurso: FUVEST
banca: FUVEST
ano: 2018
numero: 31
---

\question Maria quer comprar uma TV que está sendo vendida por R\$ 1.500,00 à vista ou em 3 parcelas mensais sem juros de R\$ 500,00. O banco oferece uma aplicação que rende 1\% ao mês.

Se Maria pagar a primeira parcela e aplicar o restante, conseguirá pagar as duas parcelas seguintes exatamente.

Quanto Maria reservou para essa compra?

\begin{choices}
\choice 1.450,20
\choice 1.480,20
\choice 1.485,20
\choice 1.495,20
\correctchoice 1.490,20
\end{choices}