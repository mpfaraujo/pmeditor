
---
tipo: Múltipla Escolha
dificuldade: Fácil
disciplina: Matemática
assunto: Matemática Financeira
gabarito: B
tags: [porcentagem, inflação, aumento real, variação percentual]

fonte: concurso
concurso: FUVEST
banca: FUVEST
ano: 2020
numero: 14
---
\question
Se, em 15 anos, o salário mínimo teve um aumento nominal de 300\% e a inflação foi de 100\%, é correto afirmar que o aumento real do salário mínimo, nesse período, foi de
\begin{choices}
\choice 50\%.
\correctchoice 100\%.
\choice 150\%.
\choice 200\%.
\choice 250\%.
\end{choices}

---
tipo: Múltipla Escolha
dificuldade: Média
disciplina: Matemática
assunto: Geometria Espacial
gabarito: C
tags: [comprimento de arco, circunferência, soma, aproximação]

fonte: concurso
concurso: FUVEST
banca: FUVEST
ano: 2020
numero: 15
---
\question
O cilindro de papelão central de uma fita crepe tem raio externo de 3 cm. A fita tem espessura de 0,01 cm e dá 100 voltas completas. Considerando que, a cada volta, o raio externo do rolo é aumentado no valor da espessura da fita, o comprimento total da fita é de, aproximadamente,
(Considere $\pi \approx 3{,}14$.)
\begin{choices}
\choice 9,4 m.
\choice 11,0 m.
\correctchoice 18,8 m.
\choice 22,0 m.
\choice 25,1 m.
\end{choices}

---
tipo: Múltipla Escolha
dificuldade: Média
disciplina: Matemática
assunto: Geometria Plana
gabarito: C
tags: [paralelogramo, área, seno, ângulo]

fonte: concurso
concurso: FUVEST
banca: FUVEST
ano: 2020
numero: 16
---
\question
Um objeto é formado por 4 hastes rígidas conectadas em seus extremos por articulações, cujos centros são os vértices de um paralelogramo. As hastes movimentam-se de tal forma que o paralelogramo permanece sempre no mesmo plano. A cada configuração desse objeto, associa-se $\theta$, a medida do menor ângulo interno do paralelogramo. A área da região delimitada pelo paralelogramo quando $\theta = 90^\circ$ é $A$.

Para que a área da região delimitada pelo paralelogramo seja $\dfrac{A}{2}$, o valor de $\theta$ é, necessariamente, igual a
\begin{choices}
\choice $15^\circ$.
\choice $22{,}5^\circ$.
\correctchoice $30^\circ$.
\choice $45^\circ$.
\choice $60^\circ$.
\end{choices}

---
tipo: Múltipla Escolha
dificuldade: Fácil
disciplina: Matemática
assunto: Geometria Espacial
gabarito: A
tags: [esfera, paralelepípedo, diagonal espacial, raio]

fonte: concurso
concurso: FUVEST
banca: FUVEST
ano: 2020
numero: 17
---
\question
A menor esfera na qual um paralelepípedo reto-retângulo de medidas $7\,\text{cm} \times 4\,\text{cm} \times 4\,\text{cm}$ está inscrito tem diâmetro de
\begin{choices}
\correctchoice 9 cm.
\choice 10 cm.
\choice 11 cm.
\choice 12 cm.
\choice 15 cm.
\end{choices}

---
tipo: Múltipla Escolha
dificuldade: Média
disciplina: Matemática
assunto: Função Quadrática
gabarito: D
tags: [máximo, receita, modelagem, função do segundo grau]

fonte: concurso
concurso: FUVEST
banca: FUVEST
ano: 2020
numero: 18
---
\question
A dona de uma lanchonete observou que, vendendo um combo a R\$ 10,00, 200 deles são vendidos por dia, e que, para cada redução de R\$ 1,00 nesse preço, ela vende 100 combos a mais. Nessas condições, qual é a máxima arrecadação diária que ela espera obter com a venda desse combo?
\begin{choices}
\choice R\$ 2.000,00
\choice R\$ 3.200,00
\choice R\$ 3.600,00
\correctchoice R\$ 4.000,00
\choice R\$ 4.800,00
\end{choices}

---
tipo: Múltipla Escolha
dificuldade: Média
disciplina: Matemática
assunto: Teoria dos Números
gabarito: C
tags: [função totiente, máximo divisor comum, aritmética]

fonte: concurso
concurso: FUVEST
banca: FUVEST
ano: 2020
numero: 19
---
\question
A função $\varphi$ de Euler determina, para cada número natural $n$, a quantidade de números naturais menores do que $n$ cujo máximo divisor comum com $n$ é igual a 1. Por exemplo, $\varphi(6)=2$, pois os números menores do que 6 com tal propriedade são 1 e 5.

Qual o valor máximo de $\varphi(n)$, para $n$ de 20 a 25?
\begin{choices}
\choice 19
\choice 20
\correctchoice 22
\choice 24
\choice 25
\end{choices}

---
tipo: Múltipla Escolha
dificuldade: Média
disciplina: Matemática
assunto: Polinômios
gabarito: B
tags: [identidade polinomial, comparação de coeficientes, expansão]

fonte: concurso
concurso: FUVEST
banca: FUVEST
ano: 2020
numero: 20
---
\question
Se $3x^{2}-9x+7 = (x-a)^{3} - (x-b)^{3}$, para todo número real $x$, o valor de $a+b$ é
\begin{choices}
\choice 3.
\correctchoice 5.
\choice 6.
\choice 9.
\choice 12.
\end{choices}

---
tipo: Múltipla Escolha
dificuldade: Média
disciplina: Matemática
assunto: Sistemas Lineares
gabarito: D
tags: [equações lineares, modelagem algébrica]

fonte: concurso
concurso: FUVEST
banca: FUVEST
ano: 2020
numero: 21
---
\question
Uma agência de turismo vendeu um total de 78 passagens para os destinos: Lisboa, Paris e Roma. Sabe-se que o número de passagens vendidas para Paris foi o dobro do número de passagens vendidas para os outros dois destinos conjuntamente. Sabe-se também que, para Roma, foram vendidas duas passagens a mais que a metade das vendidas para Lisboa.

Qual foi o total de passagens vendidas, conjuntamente, para Paris e Roma?
\begin{choices}
\choice 26
\choice 38
\choice 42
\correctchoice 62
\choice 68
\end{choices}

---
tipo: Múltipla Escolha
dificuldade: Média
disciplina: Matemática
assunto: Geometria Analítica
gabarito: B
tags: [circunferência, distância aos eixos, sistema]

fonte: concurso
concurso: FUVEST
banca: FUVEST
ano: 2020
numero: 22
---
\question
Um ponto $(x,y)$ do plano cartesiano pertence ao conjunto $F$ se é equidistante dos eixos $OX$ e $OY$ e pertence ao círculo de equação
$x^{2}+y^{2}-2x-6y+2=0$.
É correto afirmar que $F$
\begin{choices}
\choice é um conjunto vazio.
\correctchoice tem exatamente 2 pontos, um no primeiro quadrante e outro no segundo quadrante.
\choice tem exatamente 2 pontos, ambos no primeiro quadrante.
\choice tem exatamente 3 pontos, sendo dois no primeiro quadrante e outro no segundo quadrante.
\choice tem exatamente 4 pontos, sendo dois no primeiro quadrante e dois no segundo quadrante.
\end{choices}

---
tipo: Múltipla Escolha
dificuldade: Fácil
disciplina: Matemática
assunto: Média Ponderada
gabarito: D
tags: [pluviometria, média ponderada, proporção, área]

fonte: concurso
concurso: FUVEST
banca: FUVEST
ano: 2020
numero: 23
---
\question
Uma cidade é dividida em dois Setores: o Setor Sul, com área de $10\,\text{km}^2$, e o Setor Norte, com área de $30\,\text{km}^2$. Após um final de semana, foram divulgados os seguintes totais pluviométricos:

sábado: Sul 7 mm, Norte 11 mm  
domingo: Sul 9 mm, Norte 17 mm

É correto afirmar que o total pluviométrico desse final de semana na cidade inteira foi de
\begin{choices}
\choice 15 mm.
\choice 17 mm.
\choice 22 mm.
\correctchoice 25 mm.
\choice 28 mm.
\end{choices}

---
tipo: Múltipla Escolha
dificuldade: Média
disciplina: Matemática
assunto: Equações
gabarito: A
tags: [equação, segmentos colineares, resolução algébrica]

fonte: concurso
concurso: FUVEST
banca: FUVEST
ano: 2020
numero: 24
---
\question
As possíveis soluções, em polegadas (inches), para o problema matemático proposto no quadrinho, no caso em que os pontos $A$, $B$ e $C$ estão em uma mesma reta, são
\begin{choices}
\correctchoice $\dfrac{10}{3}$ e $10$.
\choice $\dfrac{10}{3}$, $5$ e $10$.
\choice $\dfrac{5}{3}$, $\dfrac{10}{3}$ e $10$.
\choice $\dfrac{5}{3}$ e $10$.
\choice $\dfrac{10}{3}$ e $5$.
\end{choices}
