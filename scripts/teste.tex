\question Uma população de bactérias é modelada por $P(t)=500\cdot 2^t$, onde $t$ é o tempo em horas. Qual é a população após 3 horas?
\begin{choices}
\choice 3000
\correctchoice 4000
\choice 4500
\choice 3500
\choice 2500
\end{choices}

\question O valor de um investimento é dado por $V(t)=2000\cdot(1{,}1)^t$. Qual será o valor após 2 anos?
\begin{choices}
\choice 2420
\choice 2200
\correctchoice 2420
\choice 2400
\choice 2600
\end{choices}

\question A função $f(x)=3^x$ vale:
\begin{choices}
\correctchoice 1
\choice 0
\choice 3
\choice -1
\choice 9
\end{choices}
para $x=0$.

\question A função $f(x)=5\cdot 2^x$ tem valor:
\begin{choices}
\choice 20
\choice 15
\choice 30
\correctchoice 40
\choice 25
\end{choices}
para $x=3$.

\question Um medicamento reduz sua concentração segundo $C(t)=100\cdot(0{,}8)^t$. Qual é a concentração após 1 hora?
\begin{choices}
\choice 75
\correctchoice 80
\choice 85
\choice 70
\choice 90
\end{choices}

\question A função $f(x)=2^{x+1}$ é equivalente a:
\begin{choices}
\choice $2^x+1$
\choice $2x+1$
\choice $2^{x}-1$
\correctchoice $2\cdot2^x$
\choice $2^{x-1}$
\end{choices}

\question Em um laboratório, a massa de uma substância dobra a cada hora. Se inicialmente há 50 g, qual é a função que modela a massa após $t$ horas?
\begin{choices}
\correctchoice $M(t)=50\cdot2^t$
\choice $M(t)=2\cdot50^t$
\choice $M(t)=50+t^2$
\choice $M(t)=50+2t$
\choice $M(t)=50\cdot t^2$
\end{choices}

\question A função $f(x)=4^x$ é crescente porque:
\begin{choices}
\choice o expoente é negativo
\choice a base é menor que 1
\correctchoice a base é maior que 1
\choice o expoente é par
\choice o domínio é positivo
\end{choices}

\question O valor de $3^2\cdot3^3$ é:
\begin{choices}
\choice 81
\choice 27
\correctchoice 243
\choice 729
\choice 9
\end{choices}

\question Uma população é dada por $P(t)=1000\cdot(0{,}9)^t$. O que ocorre com o tempo?
\begin{choices}
\choice Cresce linearmente
\choice Permanece constante
\correctchoice Diminui exponencialmente
\choice Cresce exponencialmente
\choice Oscila
\end{choices}

\question A função $f(x)=a^x$ é decrescente quando:
\begin{choices}
\choice $a>1$
\correctchoice $0<a<1$
\choice $a=1$
\choice $a<0$
\choice $a=0$
\end{choices}

\question O valor de $2^{-3}$ é:
\begin{choices}
\choice 8
\correctchoice $\frac{1}{8}$
\choice -8
\choice $\frac{1}{4}$
\choice 0
\end{choices}

\question A função $f(x)=10\cdot(1{,}05)^x$ representa:
\begin{choices}
\choice decrescimento exponencial
\choice crescimento linear
\correctchoice crescimento exponencial
\choice função quadrática
\choice função constante
\end{choices}

\question Um capital de R\$ 1000 é aplicado a juros compostos de 10\% ao ano. Após 1 ano, o valor será:
\begin{choices}
\choice 1000
\choice 1010
\choice 1050
\correctchoice 1100
\choice 1200
\end{choices}

\question A equação $2^x=8$ tem solução:
\begin{choices}
\choice 2
\choice 4
\correctchoice 3
\choice 1
\choice 8
\end{choices}

\question A função $f(x)=7\cdot3^x$ tem intercepto no eixo $y$ igual a:
\begin{choices}
\correctchoice 7
\choice 3
\choice 21
\choice 0
\choice 10
\end{choices}

\question O gráfico de $f(x)=0{,}5^x$ é:
\begin{choices}
\choice crescente
\correctchoice decrescente
\choice constante
\choice parabólico
\choice linear
\end{choices}

\question A função $f(x)=2^x+1$ tem imagem sempre:
\begin{choices}
\choice menor que 0
\choice menor que 1
\correctchoice maior que 1
\choice igual a 1
\choice igual a 0
\end{choices}

\question Se $5^x=25$, então:
\begin{choices}
\choice $x=3$
\correctchoice $x=2$
\choice $x=1$
\choice $x=4$
\choice $x=5$
\end{choices}

\question A quantidade de uma cultura triplica a cada dia. Se inicia com 2 unidades, após 2 dias haverá:
\begin{choices}
\choice 12
\choice 6
\correctchoice 18
\choice 9
\choice 8
\end{choices}
