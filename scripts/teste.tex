---
tipo: Múltipla Escolha
dificuldade: Fácil
disciplina: Matemática
assunto: Médias
gabarito: C
tags: [gráfico, estatística, notas, frequência]
fonte: concurso
concurso:
  banca: UERJ
  ano: 2025
  numero: 28
---
\question Observe no gráfico a distribuição de frequências das notas de matemática de 20 alunos de uma turma.


A média aritmética das notas de matemática de todos os alunos dessa turma é:
\begin{choices}
    \choice 6,2
    \choice 6,4
    \choice 6,6
    \choice 6,8
\end{choices}

---
tipo: Múltipla Escolha
dificuldade: Fácil
disciplina: Matemática
assunto: Função Quadrática
gabarito: A
tags: [parábola, eixos coordenados, pontos de interseção, gráfico]
fonte: concurso
concurso:
  banca: UERJ
  ano: 2025
  numero: 29
---
\question A parábola representada a seguir intersecta os eixos coordenados nos pontos $A(1,0),$ $B(-1,0)$ e $C(0,-1)$.


Essa parábola é o gráfico da função quadrática f definida pela seguinte sentença:
\begin{choices}
    \choice $f(x)=x^{2}-1$
    \choice $f(x)=x^{2}+1$
    \choice $f(x)=-x^{2}-1$
    \choice $f(x)=-x^{2}+1$
\end{choices}

---
tipo: Múltipla Escolha
dificuldade: Média
disciplina: Matemática
assunto: Porcentagem
gabarito: C
tags: [lucro, custo, venda, lojista]
fonte: concurso
concurso:
  banca: UERJ
  ano: 2025
  numero: 30
---
\question Um lojista comprou 625 camisas e vendeu cada uma delas por x reais. O custo de sua compra foi igual ao valor exato da venda de 500 camisas.
Com a venda de todas as camisas, o percentual de lucro obtido pelo lojista, em relação ao custo, foi:
\begin{choices}
    \choice 15\%
    \choice 20\%
    \choice 25\%
    \choice 30\%
\end{choices}

---
tipo: Múltipla Escolha
dificuldade: Fácil
disciplina: Matemática
assunto: Trigonometria
gabarito: C
tags: [seno, cosseno, círculo trigonométrico, segundo quadrante]
fonte: concurso
concurso:
  banca: UERJ
  ano: 2025
  numero: 31
---
\question Considere o seno e o cosseno de um ângulo $\alpha$ do segundo quadrante do círculo trigonométrico representado a seguir.


Se sen $\alpha = \frac{3}{4}$, o valor de cos $\alpha$ é:
\begin{choices}
    \choice $\frac{\sqrt{7}}{4}$
    \choice $\frac{1}{4}$
    \choice $-\frac{\sqrt{7}}{4}$
    \choice $-\frac{1}{4}$
\end{choices}

---
tipo: Múltipla Escolha
dificuldade: Média
disciplina: Matemática
assunto: Área das Figuras Planas
gabarito: A
tags: [retângulo, triângulo, área, geometria plana]
fonte: concurso
concurso:
  banca: UERJ
  ano: 2025
  numero: 32
---
\question Observe na figura o retângulo PQRS com lados $\overline{PQ}=8$ cm e $\overline{PS}=3$ cm. O ponto T é a interseção do lado RS com o segmento PU, sendo $\overline{TS}=5$ cm. O ponto U, que define o triângulo RUT, pertence à reta QR.


A área do triângulo RUT, em $cm^{2}$, é igual a:
\begin{choices}
    \choice $\frac{27}{10}$
    \choice $\frac{25}{7}$
    \choice $\frac{13}{10}$
    \choice $\frac{11}{7}$
\end{choices}

---
tipo: Múltipla Escolha
dificuldade: Média
disciplina: Matemática
assunto: Geometria Espacial
gabarito: D
tags: [tetraedro, volume, triângulo retângulo, círculo inscrito]
fonte: concurso
concurso:
  banca: UERJ
  ano: 2025
  numero: 33
---
\question O tetraedro ABCD tem altura AD e base ABC inscrita em um círculo de diâmetro AB, conforme mostra a ilustração. Sabe-se que todo triângulo inscrito em um círculo, que tem um dos lados igual ao diâmetro, é um triângulo retângulo.


Considere que $\overline{AC}=6$ cm, $\overline{BC}=8$ cm e $\overline{BD}=10\sqrt{2}$.
O volume desse tetraedro, em $cm^{3}$, é igual a:
\begin{choices}
    \choice 240
    \choice 120
    \choice 90
    \choice 80
\end{choices}

---
tipo: Múltipla Escolha
dificuldade: Fácil
disciplina: Matemática
assunto: Probabilidade Condicional
gabarito: D
tags: [curso superior, idade, probabilidade, grupo de pessoas]
fonte: concurso
concurso:
  banca: UERJ
  ano: 2025
  numero: 34
---
\question Em um grupo de 40 pessoas adultas, 18 têm mais de 50 anos e 25 têm curso superior. Dentre aquelas com mais de 50, há 8 que não têm curso superior.
Se uma pessoa escolhida ao acaso tem curso superior, a probabilidade de ela ter mais de 50 anos é:
\begin{choices}
    \choice $\frac{3}{4}$
    \choice $\frac{1}{4}$
    \choice $\frac{3}{5}$
    \choice $\frac{2}{5}$
\end{choices}