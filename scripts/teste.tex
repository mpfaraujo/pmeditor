\documentclass[12pt]{exam}
\usepackage[utf8]{inputenc}
\usepackage[T1]{fontenc}
\usepackage[brazil]{babel}
\usepackage{amsmath,amssymb}
\usepackage{graphicx}
\usepackage{enumitem}

\begin{document}

\title{CPAEN/2023 -- Matemática (Questões 1 a 22)}
\date{}
\maketitle

\begin{questions}
\section*{Análise Combinatória}
\begin{verbatim}
---
tipo: Múltipla Escolha        # Múltipla Escolha | V/F | Discursiva
dificuldade: Média             # Fácil | Média | Difícil
disciplina: Matemática
assunto: Análise Combinatória
gabarito: C                    # Letra: A-E ou V/F (deixe vazio se for discursiva)
tags: [Escola Naval, 2023]                       # Ex: [ENEM, 2024, geometria]

# Fonte (opcional — preencher se for questão de concurso)
fonte: concurso                # original | concurso
concurso: Escola Naval
banca: Marinha do Brasil
ano: 2023
numero: 1
---
\end{verbatim}

\question
A camisa de um grande clube de futebol carioca e mundial é formada por sete listras verticais (frente da camisa) das cores preta e branca, conforme a figura 1.

A empresa que confecciona a camisa oferece modelos com diferentes maneiras de disposição das cores das listras, e o clube exige que sempre exista a cor preta entre brancas e/ou a cor branca entre pretas, conforme apresentado na figura 1 (camisa original) e nas figuras 2, 3 e 4.

Considerando apenas a frente da camisa e cumprindo a exigência do clube, quantos modelos de camisa podem ser confeccionados pela empresa?

\begin{choices}
\choice 126
\choice 122
\choice 114
\choice 112
\choice 64
\end{choices}

\begin{verbatim}
---
tipo: Múltipla Escolha        # Múltipla Escolha | V/F | Discursiva
dificuldade: Média             # Fácil | Média | Difícil
disciplina: Matemática
assunto: Análise Combinatória
gabarito: A                    # Letra: A-E ou V/F (deixe vazio se for discursiva)
tags: [Escola Naval, 2023]                       # Ex: [ENEM, 2024, geometria]

# Fonte (opcional — preencher se for questão de concurso)
fonte: concurso                # original | concurso
concurso: Escola Naval
banca: Marinha do Brasil
ano: 2023
numero: 2
---
\end{verbatim}

\question
Um jogador, cansado de ganhar pouco dinheiro em jogos on-line, resolveu investir seus 20 mil reais aplicando em até 4 investimentos distintos possíveis: Poupança, Mercado de Ações, Títulos do Tesouro Nacional e Mercado Imobiliário. Considere que cada aplicação deve ser realizada em unidades de mil reais.

Assinale a opção que apresenta a quantidade de maneiras distintas de distribuir os 20 mil reais entre os investimentos.

\begin{choices}
\choice 1771
\choice 4772
\choice 10626
\choice 13781
\choice 15852
\end{choices}

\section*{Logaritmos}
\begin{verbatim}
---
tipo: Múltipla Escolha        # Múltipla Escolha | V/F | Discursiva
dificuldade: Média             # Fácil | Média | Difícil
disciplina: Matemática
assunto: Logaritmos
gabarito: B                    # Letra: A-E ou V/F (deixe vazio se for discursiva)
tags: [Escola Naval, 2023]                       # Ex: [ENEM, 2024, geometria]

# Fonte (opcional — preencher se for questão de concurso)
fonte: concurso                # original | concurso
concurso: Escola Naval
banca: Marinha do Brasil
ano: 2023
numero: 3
---
\end{verbatim}

\question
Sabendo que o valor de $\log 630 = a$ e $\log 1524 = b$, assinale a opção que apresenta o valor de $\log 1260$ em função de $a$ e $b$.

\begin{choices}
\choice $\dfrac{ab+b-1}{ab-a+1}$
\choice $\dfrac{2ab+2a-1}{ab+b+1}$
\choice $\dfrac{b+3-ab}{ab-1}$
\choice $\dfrac{2a-b-2-ab}{ab-1}$
\choice $\dfrac{1+a+b}{2+a}$
\end{choices}

\section*{Progressões}
\begin{verbatim}
---
tipo: Múltipla Escolha        # Múltipla Escolha | V/F | Discursiva
dificuldade: Média             # Fácil | Média | Difícil
disciplina: Matemática
assunto: Progressões
gabarito: E                    # Letra: A-E ou V/F (deixe vazio se for discursiva)
tags: [Escola Naval, 2023]                       # Ex: [ENEM, 2024, geometria]

# Fonte (opcional — preencher se for questão de concurso)
fonte: concurso                # original | concurso
concurso: Escola Naval
banca: Marinha do Brasil
ano: 2023
numero: 4
---
\end{verbatim}

\question
Dois aspirantes A e B correm em uma pequena pista circular, e os tempos em cada volta completa são registrados em segundos (s).

O aspirante A completou a 1ª volta em 7s, as duas primeiras voltas em 20s, as três primeiras voltas em 33s, as quatro primeiras voltas em 46s e assim sucessivamente, mantendo a regularidade.

O aspirante B completou a 1ª volta em 15s, as duas primeiras voltas em 34s, as três primeiras voltas em 53s, as quatro primeiras voltas em 72s e assim sucessivamente, mantendo também a regularidade.

Considere que os aspirantes começaram a corrida no mesmo instante e no mesmo ponto de partida. Calcule em quantos segundos, após a partida, os aspirantes se encontrarão pela 4ª vez no ponto de partida.

\begin{choices}
\choice 779
\choice 798
\choice 805
\choice 806
\choice 813
\end{choices}

\section*{Funções Polinomiais}
\begin{verbatim}
---
tipo: Múltipla Escolha        # Múltipla Escolha | V/F | Discursiva
dificuldade: Difícil             # Fácil | Média | Difícil
disciplina: Matemática
assunto: Funções Polinomiais / Interpolação
gabarito: A                    # Letra: A-E ou V/F (deixe vazio se for discursiva)
tags: [Escola Naval, 2023]                       # Ex: [ENEM, 2024, geometria]

# Fonte (opcional — preencher se for questão de concurso)
fonte: concurso                # original | concurso
concurso: Escola Naval
banca: Marinha do Brasil
ano: 2023
numero: 5
---
\end{verbatim}

\question
Seja $f$ uma função polinomial, na variável $x$, de menor grau possível e coeficientes reais, tal que $f(1)=2$, $f(4)=3$, $f(3)=4$ e $f(2)=1$.

Assim, é correto afirmar que $f$:

\begin{choices}
\choice possui ponto de inflexão em $x=\dfrac{5}{2}$
\choice possui um ponto de máximo absoluto em $x=3$
\choice possui um ponto de mínimo absoluto em $x=1$
\choice não é contínua em $x=1$
\choice não possui máximo absoluto em $x\in[2,\infty)$
\end{choices}

\section*{Cálculo Integral}
\begin{verbatim}
---
tipo: Múltipla Escolha        # Múltipla Escolha | V/F | Discursiva
dificuldade: Média             # Fácil | Média | Difícil
disciplina: Matemática
assunto: Integrais
gabarito: B                    # Letra: A-E ou V/F (deixe vazio se for discursiva)
tags: [Escola Naval, 2023]                       # Ex: [ENEM, 2024, geometria]

# Fonte (opcional — preencher se for questão de concurso)
fonte: concurso                # original | concurso
concurso: Escola Naval
banca: Marinha do Brasil
ano: 2023
numero: 6
---
\end{verbatim}

\question
Seja $F$ uma função definida por
\[
F(x)=\int_{0}^{x} f(t)\,dt,\quad x\ge 0,
\]
onde
\[
f(t)=
\begin{cases}
t, & 0\le t<1,\\
t^{2}-1, & t\ge 1.
\end{cases}
\]
Assinale a opção que apresenta o valor de $F(3)$.

\begin{choices}
\choice $\dfrac{9}{2}$
\choice $\dfrac{43}{6}$
\choice $3$
\choice $-9$
\choice $\dfrac{13}{3}$
\end{choices}

\section*{Trigonometria}
\begin{verbatim}
---
tipo: Múltipla Escolha        # Múltipla Escolha | V/F | Discursiva
dificuldade: Média             # Fácil | Média | Difícil
disciplina: Matemática
assunto: Trigonometria (Somas)
gabarito: D                    # Letra: A-E ou V/F (deixe vazio se for discursiva)
tags: [Escola Naval, 2023]                       # Ex: [ENEM, 2024, geometria]

# Fonte (opcional — preencher se for questão de concurso)
fonte: concurso                # original | concurso
concurso: Escola Naval
banca: Marinha do Brasil
ano: 2023
numero: 7
---
\end{verbatim}

\question
O valor da soma $\displaystyle \sum_{k=1}^{2023}\tg(k)\cdot \tg(k+1)$ é igual a:

\begin{choices}
\choice $\dfrac{\tg(2023)}{\tg(1)}+2023$
\choice $\dfrac{\tg(2023)}{\tg(1)}-2023$
\choice $\dfrac{\tg(2024)}{\tg(1)}+2024$
\choice $\dfrac{\tg(2024)}{\tg(1)}-2024$
\choice $\tg(2023)-\tg(2024)$
\end{choices}

\section*{Geometria Analítica (Cônicas)}
\begin{verbatim}
---
tipo: Múltipla Escolha        # Múltipla Escolha | V/F | Discursiva
dificuldade: Média             # Fácil | Média | Difícil
disciplina: Matemática
assunto: Geometria Analítica (Cônicas)
gabarito:                     # Letra: A-E ou V/F (deixe vazio se for discursiva)
tags: [Escola Naval, 2023, Anulada]                       # Ex: [ENEM, 2024, geometria]

# Fonte (opcional — preencher se for questão de concurso)
fonte: concurso                # original | concurso
concurso: Escola Naval
banca: Marinha do Brasil
ano: 2023
numero: 8
---
\end{verbatim}

\question
Uma hipérbole tem os eixos transverso e conjugado contidos nos eixos coordenados e contém os pontos médios dos lados do quadrilátero, cujos vértices são as intersecções da elipse, de equação $9x^{2}+y^{2}=36$, com os eixos coordenados. Sabendo que as coordenadas de um dos focos da hipérbole é $(1,0)$, assinale a opção que apresenta uma das equações das assíntotas dessa hipérbole.

\begin{choices}
\choice $y=\sqrt{3}\,x$
\choice $y=-\sqrt{13}\,x$
\choice $y=-2\sqrt{3}\,x$
\choice $y=13x$
\choice $y=3x$
\end{choices}

\section*{Números Complexos (I)}
\begin{verbatim}
---
tipo: Múltipla Escolha        # Múltipla Escolha | V/F | Discursiva
dificuldade: Média             # Fácil | Média | Difícil
disciplina: Matemática
assunto: Números Complexos
gabarito: C                    # Letra: A-E ou V/F (deixe vazio se for discursiva)
tags: [Escola Naval, 2023]                       # Ex: [ENEM, 2024, geometria]

# Fonte (opcional — preencher se for questão de concurso)
fonte: concurso                # original | concurso
concurso: Escola Naval
banca: Marinha do Brasil
ano: 2023
numero: 9
---
\end{verbatim}

\question
Sejam $i$ a unidade imaginária e o complexo $Z$ que satisfaz a igualdade $4|Z| = |Z-1-2i|$.

A soma dos números reais que satisfazem a igualdade é igual a:

\begin{choices}
\choice $-2$
\choice $-1$
\choice $-\dfrac{2}{15}$
\choice $-\dfrac{1}{45}$
\choice $-\dfrac{1}{5}$
\end{choices}

\section*{Geometria Plana}
\begin{verbatim}
---
tipo: Múltipla Escolha        # Múltipla Escolha | V/F | Discursiva
dificuldade: Difícil             # Fácil | Média | Difícil
disciplina: Matemática
assunto: Geometria Plana (Medianas / Áreas)
gabarito: E                    # Letra: A-E ou V/F (deixe vazio se for discursiva)
tags: [Escola Naval, 2023]                       # Ex: [ENEM, 2024, geometria]

# Fonte (opcional — preencher se for questão de concurso)
fonte: concurso                # original | concurso
concurso: Escola Naval
banca: Marinha do Brasil
ano: 2023
numero: 10
---
\end{verbatim}

\question
Considere um triângulo $T_1$ de área $w$. Os comprimentos das medianas de $T_1$ são os comprimentos dos lados de um novo triângulo $T_2$; os comprimentos das medianas de $T_2$ são os comprimentos dos lados de um novo triângulo $T_3$ e assim sucessivamente.

Sendo assim, se $w=3$, é correto afirmar que a soma de todas as áreas dos $T_n$, com $n\to\infty$, é um:

\begin{choices}
\choice número racional não inteiro.
\choice quadrado perfeito.
\choice número primo.
\choice número irracional.
\choice número par.
\end{choices}

\section*{Otimização}
\begin{verbatim}
---
tipo: Múltipla Escolha        # Múltipla Escolha | V/F | Discursiva
dificuldade: Média             # Fácil | Média | Difícil
disciplina: Matemática
assunto: Otimização (Geometria Espacial)
gabarito: A                    # Letra: A-E ou V/F (deixe vazio se for discursiva)
tags: [Escola Naval, 2023]                       # Ex: [ENEM, 2024, geometria]

# Fonte (opcional — preencher se for questão de concurso)
fonte: concurso                # original | concurso
concurso: Escola Naval
banca: Marinha do Brasil
ano: 2023
numero: 11
---
\end{verbatim}

\question
Um cilindro circular reto de raio $r$ e altura $h$ possui volume igual a $300\,\text{cm}^3$. Sabendo que o cilindro possui menor área total possível, a altura $h$, em cm, é aproximadamente igual a:

\begin{choices}
\choice 7,25
\choice 6,90
\choice 6,15
\choice 3,63
\choice 2,00
\end{choices}

\section*{Exponenciais / Logaritmos e Otimização}
\begin{verbatim}
---
tipo: Múltipla Escolha        # Múltipla Escolha | V/F | Discursiva
dificuldade: Média             # Fácil | Média | Difícil
disciplina: Matemática
assunto: Equações Exponenciais / Logaritmos
gabarito: C                    # Letra: A-E ou V/F (deixe vazio se for discursiva)
tags: [Escola Naval, 2023]                       # Ex: [ENEM, 2024, geometria]

# Fonte (opcional — preencher se for questão de concurso)
fonte: concurso                # original | concurso
concurso: Escola Naval
banca: Marinha do Brasil
ano: 2023
numero: 12
---
\end{verbatim}

\question
Analise os gráficos a seguir.

Seja um número real positivo $x$, tal que $\dfrac{40}{x}=4^{x}$ e, utilizando os dados dos gráficos das funções definidas por $y=x\cdot e^{x}$ e $y=\ln(x)$, é correto afirmar que o valor de $x$ é aproximadamente igual a:

\begin{choices}
\choice 1,60
\choice 1,96
\choice 2,13
\choice 2,24
\choice 2,52
\end{choices}

\begin{verbatim}
---
tipo: Múltipla Escolha        # Múltipla Escolha | V/F | Discursiva
dificuldade: Difícil             # Fácil | Média | Difícil
disciplina: Matemática
assunto: Otimização (Mínimos)
gabarito: A                    # Letra: A-E ou V/F (deixe vazio se for discursiva)
tags: [Escola Naval, 2023]                       # Ex: [ENEM, 2024, geometria]

# Fonte (opcional — preencher se for questão de concurso)
fonte: concurso                # original | concurso
concurso: Escola Naval
banca: Marinha do Brasil
ano: 2023
numero: 13
---
\end{verbatim}

\question
Se $x>0$, seja $f(x)=5x^{2}+Ax^{-5}$, em que $A$ é uma constante positiva. Assinale a opção que apresenta o menor valor de $A$ tal que $f(x)\ge 24$, $\forall x>0$.

\begin{choices}
\choice $\dfrac{2^{11}\cdot 3^{3}}{7^{3}}\sqrt{\dfrac{6}{7}}$
\choice $\dfrac{2^{3}\cdot 3^{11}}{7^{11}}\sqrt{\dfrac{7}{6}}$
\choice $2^{8}$
\choice $\dfrac{2^{7}\cdot 3^{11}}{7^{11}}\sqrt{\dfrac{2}{3}}$
\choice $\dfrac{2^{8}\cdot 3^{-7}}{7^{-3}}\sqrt{\dfrac{3}{2}}$
\end{choices}

\section*{Lógica Proposicional}
\begin{verbatim}
---
tipo: Múltipla Escolha        # Múltipla Escolha | V/F | Discursiva
dificuldade: Média             # Fácil | Média | Difícil
disciplina: Matemática
assunto: Lógica Proposicional
gabarito: B                    # Letra: A-E ou V/F (deixe vazio se for discursiva)
tags: [Escola Naval, 2023]                       # Ex: [ENEM, 2024, geometria]

# Fonte (opcional — preencher se for questão de concurso)
fonte: concurso                # original | concurso
concurso: Escola Naval
banca: Marinha do Brasil
ano: 2023
numero: 14
---
\end{verbatim}

\question
Na Escola Naval, um oficial afirmou que: ``Se todos os navios estão fundeados, então hoje é dia de licença pagamento ou os nautas não estão de serviço''.

Assinale a opção que apresenta a afirmativa que equivale à afirmação do oficial.

\begin{choices}
\choice Se todos os navios não estão fundeados, então hoje não é dia de licença pagamento ou os nautas estão de serviço.
\choice Se hoje não é dia de licença pagamento e os nautas estão de serviço, então há navio que não está fundeado.
\choice Se hoje é dia de licença pagamento ou os nautas não estão de serviço, então todos os navios estão fundeados.
\choice Se hoje não é dia de licença pagamento ou os nautas estão de serviço, então há navio que não está fundeado.
\choice Se hoje é dia de licença pagamento e os nautas estão de serviço, então há navio que não está fundeado.
\end{choices}

\section*{Geometria Espacial}
\begin{verbatim}
---
tipo: Múltipla Escolha        # Múltipla Escolha | V/F | Discursiva
dificuldade: Difícil             # Fácil | Média | Difícil
disciplina: Matemática
assunto: Geometria Espacial
gabarito: B                    # Letra: A-E ou V/F (deixe vazio se for discursiva)
tags: [Escola Naval, 2023]                       # Ex: [ENEM, 2024, geometria]

# Fonte (opcional — preencher se for questão de concurso)
fonte: concurso                # original | concurso
concurso: Escola Naval
banca: Marinha do Brasil
ano: 2023
numero: 15
---
\end{verbatim}

\question
Analise a figura a seguir.

Seja o paralelepípedo reto retângulo $ABCDEFGH$ com medidas $AB=4\text{ cm}$, $AD=8\text{ cm}$ e $AE=5\text{ cm}$, conforme a figura. Considerando o ponto $P$ como o centro do paralelepípedo, é correto afirmar que a distância, em cm, de $P$ ao plano $BDE$ é igual a:

\begin{choices}
\choice $\dfrac{10}{\sqrt{189}}$
\choice $\dfrac{20}{\sqrt{189}}$
\choice $\dfrac{35}{2\sqrt{105}}$
\choice $\dfrac{30}{\sqrt{189}}$
\choice $\dfrac{35}{\sqrt{105}}$
\end{choices}

\section*{Geometria Analítica (Sistemas)}
\begin{verbatim}
---
tipo: Múltipla Escolha        # Múltipla Escolha | V/F | Discursiva
dificuldade: Difícil             # Fácil | Média | Difícil
disciplina: Matemática
assunto: Geometria Analítica (Sistemas)
gabarito: A                    # Letra: A-E ou V/F (deixe vazio se for discursiva)
tags: [Escola Naval, 2023]                       # Ex: [ENEM, 2024, geometria]

# Fonte (opcional — preencher se for questão de concurso)
fonte: concurso                # original | concurso
concurso: Escola Naval
banca: Marinha do Brasil
ano: 2023
numero: 16
---
\end{verbatim}

\question
Considere que um jogador do jogo eletrônico \emph{Call of Duty Warzone}, sabedor de matemática, foi questionado por seus amigos sobre quantas unidades de área da região do campo de batalha devem ser vasculhadas para encontrar os últimos inimigos a serem derrubados para que, assim, eles vençam a partida.

Supondo que a região do campo de batalha seja totalmente plana, o valor da área que deve ser vasculhada é igual ao valor da área do polígono convexo cujos vértices são os pares cartesianos da solução do sistema
\[
\begin{cases}
x^{2}=13x+4y\\
y^{2}=4x+13y
\end{cases}
\]
Assinale a opção que apresenta o valor da área dessa região.

\begin{choices}
\choice 255
\choice 260
\choice 265
\choice 270
\choice 275
\end{choices}

\section*{Geometria Espacial (Volumes)}
\begin{verbatim}
---
tipo: Múltipla Escolha        # Múltipla Escolha | V/F | Discursiva
dificuldade: Difícil             # Fácil | Média | Difícil
disciplina: Matemática
assunto: Geometria Espacial (Volumes)
gabarito: D                    # Letra: A-E ou V/F (deixe vazio se for discursiva)
tags: [Escola Naval, 2023]                       # Ex: [ENEM, 2024, geometria]

# Fonte (opcional — preencher se for questão de concurso)
fonte: concurso                # original | concurso
concurso: Escola Naval
banca: Marinha do Brasil
ano: 2023
numero: 17
---
\end{verbatim}

\question
Na partida de futebol entre Alemanha e Japão na Copa do Mundo de 2022, no lance que originou o segundo gol do Japão, por 1,88\,mm, a bola não passou totalmente pela linha de fundo (linha branca), como mostram as figuras 1, 2 e 3.

Sabe-se que a parte da bola sobre a linha branca equivale a uma calota esférica de altura 1,88\,mm e volume $V_c$. Considerando a bola de futebol uma esfera de raio 70\,cm e volume $V$, assinale a opção que apresenta corretamente a relação entre $V$ e $V_c$.

\begin{choices}
\choice $0,1\%\ \text{de }V < V_c < 1\%\ \text{de }V$
\choice $0,01\%\ \text{de }V < V_c < 0,1\%\ \text{de }V$
\choice $0,001\%\ \text{de }V < V_c < 0,01\%\ \text{de }V$
\choice $0,0001\%\ \text{de }V < V_c < 0,001\%\ \text{de }V$
\choice $V_c < 0,0001\%$
\end{choices}

\section*{Limites}
\begin{verbatim}
---
tipo: Múltipla Escolha        # Múltipla Escolha | V/F | Discursiva
dificuldade: Difícil             # Fácil | Média | Difícil
disciplina: Matemática
assunto: Limites
gabarito: E                    # Letra: A-E ou V/F (deixe vazio se for discursiva)
tags: [Escola Naval, 2023]                       # Ex: [ENEM, 2024, geometria]

# Fonte (opcional — preencher se for questão de concurso)
fonte: concurso                # original | concurso
concurso: Escola Naval
banca: Marinha do Brasil
ano: 2023
numero: 18
---
\end{verbatim}

\question
Assinale a opção que apresenta o resultado de
\[
\lim_{x\to+\infty}\left(\frac{3x+9}{4x+1}\right)^{4x+4}.
\]

\begin{choices}
\choice $+\infty$
\choice $-\infty$
\choice $e^{-1}$
\choice $e^{-\frac{3}{4}}$
\choice $0$
\end{choices}

\section*{Geometria Espacial (Reta e Plano)}
\begin{verbatim}
---
tipo: Múltipla Escolha        # Múltipla Escolha | V/F | Discursiva
dificuldade: Média             # Fácil | Média | Difícil
disciplina: Matemática
assunto: Geometria Espacial (Reta e Plano)
gabarito:                     # Letra: A-E ou V/F (deixe vazio se for discursiva)
tags: [Escola Naval, 2023, Anulada]                       # Ex: [ENEM, 2024, geometria]

# Fonte (opcional — preencher se for questão de concurso)
fonte: concurso                # original | concurso
concurso: Escola Naval
banca: Marinha do Brasil
ano: 2023
numero: 19
---
\end{verbatim}

\question
Seja a reta $r$ oblíqua a um plano $\varphi$ pelo ponto $A$. Considere que a reta $s\subset\varphi$ é a projeção ortogonal de $r$ sobre $\varphi$ e a reta $t\subset\varphi$ que passa por $A$. Se $\theta$ é o ângulo entre $r$ e $s$, e $\beta$ é o ângulo entre $r$ e $t$, assinale a opção que apresenta a correta relação entre esses ângulos.

\begin{choices}
\choice $\theta<\beta$
\choice $\theta=\beta$
\choice $\beta<\theta$
\choice $\theta=2\beta$
\choice $2\beta<\theta$
\end{choices}

\section*{Matrizes e Determinantes}
\begin{verbatim}
---
tipo: Múltipla Escolha        # Múltipla Escolha | V/F | Discursiva
dificuldade: Média             # Fácil | Média | Difícil
disciplina: Matemática
assunto: Matrizes e Determinantes
gabarito: B                    # Letra: A-E ou V/F (deixe vazio se for discursiva)
tags: [Escola Naval, 2023]                       # Ex: [ENEM, 2024, geometria]

# Fonte (opcional — preencher se for questão de concurso)
fonte: concurso                # original | concurso
concurso: Escola Naval
banca: Marinha do Brasil
ano: 2023
numero: 20
---
\end{verbatim}

\question
Sejam as matrizes $A$ e $B$, ambas de ordem $4\times 4$, com $\det A=-1$ e $\det B=2$. Calcule o determinante da matriz $X$ sabendo que
\[
A^{-1}\cdot B^{T}\cdot X = 2A^{T}
\]
e assinale a opção correta.

\begin{choices}
\choice $-8$
\choice $8$
\choice $-16$
\choice $16$
\choice $32$
\end{choices}

\section*{Números Complexos (II)}
\begin{verbatim}
---
tipo: Múltipla Escolha        # Múltipla Escolha | V/F | Discursiva
dificuldade: Média             # Fácil | Média | Difícil
disciplina: Matemática
assunto: Números Complexos (Equações)
gabarito: E                    # Letra: A-E ou V/F (deixe vazio se for discursiva)
tags: [Escola Naval, 2023]                       # Ex: [ENEM, 2024, geometria]

# Fonte (opcional — preencher se for questão de concurso)
fonte: concurso                # original | concurso
concurso: Escola Naval
banca: Marinha do Brasil
ano: 2023
numero: 21
---
\end{verbatim}

\question
Considere que parte real e imaginária de um número complexo $z$ sejam denotadas, respectivamente, por $\Re(z)$ e $\Im(z)$.

Sejam $z_1$ e $z_2$ números complexos que satisfazem a equação
\[
z^{2} + (1-i)z - 3i = 0,
\]
onde $i$ é a unidade imaginária. Sobre $z_1$ e $z_2$, é correto afirmar que:

\begin{choices}
\choice $\Im(z_1+z_2)=0$
\choice $\Re(z_1+z_2)=0$
\choice $\Im(z_1)\cdot \Im(z_2)$ é um número irracional.
\choice $\Re(z_1)\cdot \Re(z_2)$ é um número irracional.
\choice $\Im(z_1+z_2)+\Re(z_1+z_2)=0$
\end{choices}

\section*{Trigonometria e Área}
\begin{verbatim}
---
tipo: Múltipla Escolha        # Múltipla Escolha | V/F | Discursiva
dificuldade: Média             # Fácil | Média | Difícil
disciplina: Matemática
assunto: Trigonometria / Geometria (Área)
gabarito: C                    # Letra: A-E ou V/F (deixe vazio se for discursiva)
tags: [Escola Naval, 2023]                       # Ex: [ENEM, 2024, geometria]

# Fonte (opcional — preencher se for questão de concurso)
fonte: concurso                # original | concurso
concurso: Escola Naval
banca: Marinha do Brasil
ano: 2023
numero: 22
---
\end{verbatim}

\question
Considere que um navio de guerra da Marinha do Brasil danificou o radar de um navio inimigo com um tiro de canhão 50\,mm.

O radar do navio inimigo passou a detectar apenas numa região angular que é solução de
\[
\sen(x)+\cos(x)\ge \frac{\sqrt{2}}{2},
\]
sendo $x$ real.

Se o radar do navio inimigo tem um poder de alcance que está dentro da região da circunferência $x^{2}+y^{2}=R^{2}$, com $R\le 100\,\text{km}$, assinale a opção que apresenta a área total de alcance desse radar em que não será possível detectar o navio de guerra da Marinha do Brasil.

\begin{choices}
\choice $\dfrac{\pi}{3}R^{2}\ \text{km}^{2}$
\choice $\dfrac{\pi}{12}R^{2}\ \text{km}^{2}$
\choice $\dfrac{2\pi}{3}R^{2}\ \text{km}^{2}$
\choice $\dfrac{3\pi}{4}R^{2}\ \text{km}^{2}$
\choice $\dfrac{5\pi}{6}R^{2}\ \text{km}^{2}$
\end{choices}

\end{questions}

\end{document}
