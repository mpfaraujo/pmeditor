---
tipo: Múltipla Escolha
dificuldade: Média
disciplina: Matemática
assunto: Interpretação de Gráficos
gabarito: C
tags: [função, crescimento, variação, análise gráfica]

fonte: concurso
concurso: FUVEST
banca: FUVEST
ano: 2019
numero: 49
---
\question
Um dono de restaurante assim descreveu a evolução do faturamento quinzenal de seu negócio, ao longo dos dez primeiros meses após a inauguração: “Até o final dos três primeiros meses, tivemos uma velocidade de crescimento mais ou menos constante, quando então sofremos uma queda abrupta, com o faturamento caindo à metade do que tinha sido atingido. Em seguida, voltamos a crescer, igualando, um mês e meio depois dessa queda, o faturamento obtido ao final do terceiro mês. Agora, ao final do décimo mês, estamos estabilizando o faturamento em um patamar 50\% acima do faturamento obtido ao final do terceiro mês”.

Considerando que, na ordenada, o faturamento quinzenal está representado em unidades desconhecidas, porém uniformemente espaçadas, qual dos gráficos é compatível com a descrição do comerciante?

\begin{choices}
\choice (A)
\choice (B)
\correctchoice (C)
\choice (D)
\choice (E)
\end{choices}

---
tipo: Múltipla Escolha
dificuldade: Difícil
disciplina: Matemática
assunto: Progressão Geométrica
gabarito: D
tags: [potência de 2, crescimento exponencial, ordem de grandeza]

fonte: concurso
concurso: FUVEST
banca: FUVEST
ano: 2019
numero: 50
---
\question
Forma-se uma pilha de folhas de papel, em que cada folha tem $0{,}1$ mm de espessura. A pilha é formada da seguinte maneira: coloca-se uma folha na primeira vez e, em cada uma das vezes seguintes, tantas quantas já houverem sido colocadas anteriormente. Depois de 33 dessas operações, a altura da pilha terá a ordem de grandeza

\begin{choices}
\choice da altura de um poste.
\choice da altura de um prédio de 30 andares.
\choice do comprimento da Av. Paulista.
\correctchoice da distância da cidade de São Paulo (SP) à cidade do Rio de Janeiro (RJ).
\choice do diâmetro da Terra.
\end{choices}

---
tipo: Múltipla Escolha
dificuldade: Média
disciplina: Matemática
assunto: Geometria Espacial
gabarito: C
tags: [volume, soma, paralelepípedo]

fonte: concurso
concurso: FUVEST
banca: FUVEST
ano: 2019
numero: 51
---
\question
A figura mostra uma escada maciça de quatro degraus, todos eles com formato de um paralelepípedo reto-retângulo. A base de cada degrau é um retângulo de dimensões $20$ cm por $50$ cm, e a diferença de altura entre o piso e o primeiro degrau e entre os degraus consecutivos é de $10$ cm. Se essa escada for prolongada para ter 20 degraus, mantendo o mesmo padrão, seu volume será igual a

\begin{choices}
\choice $2{,}1\ \text{m}^3$
\choice $2{,}3\ \text{m}^3$
\correctchoice $3{,}0\ \text{m}^3$
\choice $4{,}2\ \text{m}^3$
\choice $6{,}0\ \text{m}^3$
\end{choices}

---
tipo: Múltipla Escolha
dificuldade: Média
disciplina: Matemática
assunto: Geometria Espacial
gabarito: D
tags: [área, geometria espacial, quadrilátero]

fonte: concurso
concurso: FUVEST
banca: FUVEST
ano: 2019
numero: 52
---
\question
Uma empresa estuda cobrir um vão entre dois prédios (com formato de paralelepípedos reto-retângulos) que têm paredes laterais paralelas, instalando uma lona na forma de um quadrilátero, com pontas presas nos pontos $A$, $B$, $C$ e $D$, conforme indicação da figura.

Sabendo que a lateral de um prédio tem $80$ m de altura e $28$ m de largura, que a lateral do outro prédio tem $60$ m de altura e $20$ m de largura e que essas duas paredes laterais distam $15$ m uma da outra, a área total dessa lona seria de

\begin{choices}
\choice $300\ \text{m}^2$
\choice $360\ \text{m}^2$
\choice $600\ \text{m}^2$
\correctchoice $720\ \text{m}^2$
\choice $1.200\ \text{m}^2$
\end{choices}

---
tipo: Múltipla Escolha
dificuldade: Média
disciplina: Matemática
assunto: Função Logarítmica
gabarito: E
tags: [logaritmo, mudança de base, exponencial]

fonte: concurso
concurso: FUVEST
banca: FUVEST
ano: 2019
numero: 53
---
\question
Se $\log_2 y = -\dfrac{1}{2} + \dfrac{2}{3}\log_2 x$, para $x>0$, então

\begin{choices}
\choice $y=\dfrac{\sqrt{x^{2}}}{\sqrt{2}}$
\choice $y=\sqrt{\dfrac{x^{3}}{2}}$
\choice $y=-\dfrac{1}{\sqrt{2}}+\sqrt{x^{2}}$
\choice $y=\sqrt{2}\,\sqrt{x^{2}}$
\correctchoice $y=\sqrt{2}\,x^{3}$
\end{choices}

---
tipo: Múltipla Escolha
dificuldade: Difícil
disciplina: Matemática
assunto: Geometria Plana
gabarito: A
tags: [rotação, arco de circunferência, trajetória]

fonte: concurso
concurso: FUVEST
banca: FUVEST
ano: 2019
numero: 54
---
\question
Um triângulo retângulo com vértices denominados $A$, $B$ e $C$ apoia-se sobre uma linha horizontal, que corresponde ao solo, e gira sem escorregar no sentido horário. Isto é, se a posição inicial é aquela mostrada na figura, o movimento começa com uma rotação em torno do vértice $C$ até o vértice $A$ tocar o solo, após o que passa a ser uma rotação em torno de $A$, até o vértice $B$ tocar o solo, e assim por diante.

Usando as dimensões indicadas na figura ($BA=1$ e $CB=2$), qual é o comprimento da trajetória percorrida pelo vértice $B$, desde a posição mostrada, até a aresta $CB$ apoiar-se no solo novamente?

\begin{choices}
\correctchoice $\dfrac{3}{2}\pi$
\choice $\dfrac{3+\sqrt{3}}{3}\pi$
\choice $\dfrac{13}{6}\pi$
\choice $\dfrac{3+\sqrt{3}}{2}\pi$
\choice $\dfrac{8+2\sqrt{3}}{3}\pi$
\end{choices}

---
tipo: Múltipla Escolha
dificuldade: Média
disciplina: Matemática
assunto: Probabilidade
gabarito: B
tags: [probabilidade, caminhada aleatória, combinatória]

fonte: concurso
concurso: FUVEST
banca: FUVEST
ano: 2019
numero: 55
---
\question
Uma seta aponta para a posição zero no instante inicial. A cada rodada, ela poderá ficar no mesmo lugar ou mover-se uma unidade para a direita ou mover-se uma unidade para a esquerda, cada uma dessas três possibilidades com igual probabilidade.

Qual é a probabilidade de que, após 5 rodadas, a seta volte à posição inicial?

\begin{choices}
\choice $\dfrac{1}{9}$
\correctchoice $\dfrac{17}{81}$
\choice $\dfrac{1}{3}$
\choice $\dfrac{51}{125}$
\choice $\dfrac{125}{243}$
\end{choices}
