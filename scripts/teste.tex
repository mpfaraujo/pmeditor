---
tipo: Múltipla Escolha
dificuldade: Média
disciplina: Matemática
assunto: Problemas sobre as 4 Operações
gabarito: C
tags: [moeda, conversão, Brás Cubas, histórico]
fonte: concurso
concurso:
  banca: FUVEST
  ano: 2016
  numero: 01
---
\question De 1869 até hoje, ocorreram as seguintes mudanças de moeda no Brasil: (1) em 1942, foi criado o cruzeiro, cada cruzeiro valendo mil réis; (2) em 1967, foi criado o cruzeiro novo, cada cruzeiro novo valendo mil cruzeiros; em 1970, o cruzeiro novo voltou a se chamar apenas cruzeiro; (3) em 1986, foi criado o cruzado, cada cruzado valendo mil cruzeiros; (4) em 1989, foi criado o cruzado novo, cada um valendo mil cruzados; em 1990, o cruzado novo passou a se chamar novamente cruzeiro; (5) em 1993, foi criado o cruzeiro real, cada um valendo mil cruzeiros; (6) em 1994, foi criado o real, cada um valendo 2.750 cruzeiros reais.

Quando morreu, em 1869, Brás Cubas possuía 300 contos. Se esse valor tivesse ficado até hoje em uma conta bancária, sem receber juros e sem pagar taxas, e se, a cada mudança de moeda, o depósito tivesse sido normalmente convertido para a nova moeda, o saldo hipotético dessa conta seria, aproximadamente, de um décimo de

\begin{choices}
    \choice real.
    \choice milésimo de real.
    \choice milionésimo de real.
    \choice bilionésimo de real.
    \choice trilionésimo de real.
\end{choices}

---
tipo: Múltipla Escolha
dificuldade: Média
disciplina: Matemática
assunto: Geometria Plana
gabarito: D
tags: [pontos, colineares, circunferência, mediatriz, triângulo]
fonte: concurso
concurso:
  banca: FUVEST
  ano: 2016
  numero: 02
---
\question Os pontos A, B e C são colineares, $AB=5$, $BC=2$ e B está entre A e C. Os pontos C e D pertencem a uma circunferência com centro em A. Traça-se uma reta $r$ perpendicular ao segmento $\overline{BD}$ passando pelo seu ponto médio. Chama-se de P a interseção de $r$ com $\overline{AD}$. Então, $AP+BP$ vale

\begin{choices}
    \choice 4
    \choice 5
    \choice 6
    \choice 7
    \choice 8
\end{choices}

---
tipo: Múltipla Escolha
dificuldade: Fácil
disciplina: Matemática
assunto: Médias
gabarito: C
tags: [velocidade média, trajetos, harmônica]
fonte: concurso
concurso:
  banca: FUVEST
  ano: 2016
  numero: 03
---
\question Um veículo viaja entre dois povoados da Serra da Mantiqueira, percorrendo a primeira terça parte do trajeto à velocidade média de $60~km/h$, a terça parte seguinte a $40~km/h$ e o restante do percurso a $20~km/h$. O valor que melhor aproxima a velocidade média do veículo nessa viagem, em $km/h$, é

\begin{choices}
    \choice 32,5
    \choice 35
    \choice 37,5
    \choice 40
    \choice 42,5
\end{choices}

---
tipo: Múltipla Escolha
dificuldade: Fácil
disciplina: Matemática
assunto: Fatoração e Produtos Notáveis
gabarito: E
tags: [identidades, álgebra, raízes]
fonte: concurso
concurso:
  banca: FUVEST
  ano: 2016
  numero: 04
---
\question A igualdade correta para quaisquer $a$ e $b$, números reais maiores do que zero, é

\begin{choices}
    \choice $\sqrt[3]{a^{3}+b^{3}}=a+b$
    \choice $\frac{1}{a-\sqrt{a^{2}+b^{2}}}=-\frac{1}{b}$
    \choice $(\sqrt{a}-\sqrt{b})^{2}=a-b$
    \choice $\frac{1}{a+b}=\frac{1}{a}+\frac{1}{b}$
    \choice $\frac{a^{3}-b^{3}}{a^{2}+ab+b^{2}}=a-b$
\end{choices}

---
tipo: Múltipla Escolha
dificuldade: Média
disciplina: Matemática
assunto: Probabilidade Simples
gabarito: C
tags: [bolas, caixa, retirada sem reposição]
fonte: concurso
concurso:
  banca: FUVEST
  ano: 2016
  numero: 05
---
\question Em um experimento probabilístico, Joana retirará aleatoriamente 2 bolas de uma caixa contendo bolas azuis e bolas vermelhas. Ao montar-se o experimento, colocam-se 6 bolas azuis na caixa. Quantas bolas vermelhas devem ser acrescentadas para que a probabilidade de Joana obter 2 azuis seja $1/3$?

\begin{choices}
    \choice 2
    \choice 4
    \choice 6
    \choice 8
    \choice 10
\end{choices}

---
tipo: Múltipla Escolha
dificuldade: Difícil
disciplina: Matemática
assunto: Geometria Analítica
gabarito: B
tags: [círculo, tangência, parábola, plano cartesiano]
fonte: concurso
concurso:
  banca: FUVEST
  ano: 2016
  numero: 06
---
\question No plano cartesiano, um círculo de centro $P=(a,b)$ tangencia as retas de equações $y=x$ e $x=0$. Se P pertence à parábola de equação $y=x^{2}$ e $a>0$, a ordenada $b$ do ponto P é igual a

\begin{choices}
    \choice $2+2\sqrt{2}$
    \choice $3+2\sqrt{2}$
    \choice $4+2\sqrt{2}$
    \choice $5+2\sqrt{2}$
    \choice $6+2\sqrt{2}$
\end{choices}

---
tipo: Múltipla Escolha
dificuldade: Média
disciplina: Matemática
assunto: Medidas de Tendência Central
gabarito: C
tags: [notas, média aritmética, alunos]
fonte: concurso
concurso:
  banca: FUVEST
  ano: 2016
  numero: 07
---
\question Em uma classe com 14 alunos, 8 são mulheres e 6 são homens. A média das notas das mulheres no final do semestre ficou 1 ponto acima da média da classe. A soma das notas dos homens foi metade da soma das notas das mulheres. Então, a média das notas dos homens ficou mais próxima de

\begin{choices}
    \choice 4,3
    \choice 4,5
    \choice 4,7
    \choice 4,9
    \choice 5,1
\end{choices}

---
tipo: Múltipla Escolha
dificuldade: Média
disciplina: Matemática
assunto: Área das Figuras Planas
gabarito: E
tags: [quadrilátero, ângulos retos, cosseno, diagonal]
fonte: concurso
concurso:
  banca: FUVEST
  ano: 2016
  numero: 08
---
\question No quadrilátero plano ABCD, os ângulos $A\hat{B}C$ e $A\hat{D}C$ são retos, $AB=AD=1$, $BC=CD=2$ e BD é uma diagonal. O cosseno do ângulo $B\hat{C}D$ vale

\begin{choices}
    \choice $\frac{\sqrt{3}}{5}$
    \choice $\frac{2}{5}$
    \choice $\frac{3}{5}$
    \choice $\frac{2\sqrt{3}}{5}$
    \choice $\frac{4}{5}$
\end{choices}

---
tipo: Múltipla Escolha
dificuldade: Média
disciplina: Matemática
assunto: Função Logarítmica
gabarito: B
tags: [simplificação, mudança de base]
fonte: concurso
concurso:
  banca: FUVEST
  ano: 2016
  numero: 09
---
\question Use as propriedades do logaritmo para simplificar a expressão
\[ S=\frac{1}{2\cdot \log_{2}2016}+\frac{1}{5\cdot \log_{3}2016}+\frac{1}{10\cdot \log_{7}2016} \]
O valor de $S$ é

\begin{choices}
    \choice $1/2$
    \choice $1/10$
    \choice $1/5$
    \choice $1/3$
    \choice $1/7$
\end{choices}

---
tipo: Múltipla Escolha
dificuldade: Difícil
disciplina: Matemática
assunto: Geometria Espacial
gabarito: A
tags: [tetraedro regular, seção plana, área]
fonte: concurso
concurso:
  banca: FUVEST
  ano: 2016
  numero: 10
---
\question Cada aresta do tetraedro regular ABCD mede 10. Por um ponto P na aresta $\overline{AC}$ passa o plano $\alpha$ paralelo às arestas $\overline{AB}$ e $\overline{CD}$. Dado que $AP=3$, o quadrilátero determinado pelas interseções de $\alpha$ com as arestas do tetraedro tem área igual a

\begin{choices}
    \choice 21
    \choice $\frac{21\sqrt{2}}{2}$
    \choice 30
    \choice $\frac{30}{2}$
    \choice $\frac{30\sqrt{3}}{2}$
\end{choices}

---
tipo: Múltipla Escolha
dificuldade: Média
disciplina: Química
assunto: Outros
gabarito: B
tags: [pH, diluição, soda cáustica]
fonte: concurso
concurso:
  banca: FUVEST
  ano: 2016
  numero: 11
---
\question Dispõe-se de 2 litros de uma solução aquosa de soda cáustica que apresenta pH 9. O volume de água, em litros, que deve ser adicionado a esses 2 litros para que a solução resultante apresente pH 8 é

\begin{choices}
    \choice 2
    \choice 18
    \choice 6
    \choice 10
    \choice 14
\end{choices}